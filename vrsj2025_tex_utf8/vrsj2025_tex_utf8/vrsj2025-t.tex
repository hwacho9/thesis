%%%%%%%%%%%%%%%%%%%%%%%%%%%%%%%%%%%%%%%%%%%%%%%%%%%%%%%%%%%%%%%%%%%%%%
%  日本バーチャルリアリティ学会大会論文集
%  大会論文集投稿用原稿執筆要領(サンプル原稿)
%
%  Apr. 10, 2013  Arranged by Megumi Nakao
%  Feb.  5, 2014  Arranged by Keita Suzuki, Aichi Institute of Technology
%  Feb. 20, 2015  Arranged by Yasuyuki Yanagida
%  Feb. 20, 2019  Arranged by Shoichi Hasegawa
%%%%%%%%%%%%%%%%%%%%%%%%%%%%%%%%%%%%%%%%%%%%%%%%%%%%%%%%%%%%%%%%%%%%%%

\documentclass[a4paper]{jarticle}
%%% 本大会論文集固有のパラメータ,定義を読み込み.
  \usepackage{vrsjj}
%%% 図貼り付け用.必要に応じて,使用環境に適合するよう編集してください.
  \usepackage[dvipdfmx]{graphicx}
%%% 最終ページの高さを自動的に揃える場合,balanceパッケージを使用可.
%%% multicolパッケージを使うと脚注が二段組でなくなるため,
%%% 脚注の仕組みを利用している英文著者の表示と干渉します.
  \usepackage{balance}

  \special{pdf: pagesize width 210truemm height 297truemm} 
  
%%% ヘッダ用定義.
\newcounter{vrsjyear}
\newcounter{vrsjmonth}
\newcounter{vrsjnum}

\setcounter{vrsjyear}{2025}
\setcounter{vrsjmonth}{9}
\setcounter{vrsjnum}{\value{vrsjyear}}
\addtocounter{vrsjnum}{-1995}

%%% 行間の指定: \baselinestrechの値が1.32で1ページ50行に相当します.
\renewcommand{\baselinestretch}{1.32}

\begin{document}%%%%%%%%%%%%%%%%%%%%%%%%%%%%%%%%%%%%%%%%%%%%%%%%%%%%%%
\small %%% フォントのサイズを small (9 pt) に設定.
\twocolumn[%%%%%%%%%%%%%%%%%%%% []内が1段組部分.
%%%【必須】
\HeadComm{This article is a technical report without peer review, and its polished and/or extended version may be published elsewhere.}%
%%% 【必須】ロゴとヘッダ.変更しないでください.
\ProcTitle{第\arabic{vrsjnum}回日本バーチャルリアリティ学会大会論文集(\arabic{vrsjyear}年\arabic{vrsjmonth}月)}%
%%%
%%% 【必須】和文タイトル.
\JTitle{VRコンテンツ体験時における生体・行動情報の統合に基づく\\リアルタイム感情推定フレームワーク}%
%%% 【オプション】英文タイトル.英文タイトルを記載する場合のみ有効にしてください.
\ETitle{A Real-Time Emotion Estimation Framework Based on Integration of Physiological and Behavioral Information During VR Content Experience}
%%% 【必須】和文著者.
\JAuthor{趙 聖化$^{1)}$}%
%%% 【オプション】英文著者.
\EAuthor{Sunghwa CHO}
%%% 【必須】和文所属.機関名と住所の間の改行はなくなりました
\Affiliation{1) 関西学院大学 人間システム工学科 井村研究室 (〒669-1330 兵庫県三田市学園上ケ原1番1号, fqd69170@kwansei.ac.jp)}%
%%% 【必須】和文要旨
\Abstract{%
本研究では,Unityで制作したランダム迷路型VRホラーゲームを用いて,体験者の生体信号の変化を統計的に分析し,集中・恐怖・驚き状態の生理的特徴を明らかにした.健康な男性6名(平均25.33歳)が参加し,手のひらのEDA,左胸部‐腸骨稜のECG,前腕筋EMG(電極間20mm)を1kHzで測定しながら,約5〜6分間ゲームをプレイした.セグメントごとにゾンビが出現した直後の平均HRV(RMSSD)は安静時と比べて−44%,EDAは+148%,EMGは+37%上昇し,恐怖刺激が交感神経系の活性化および筋緊張の増加を引き起こすことが確認された. 結果分析では,VRの恐怖刺激時に心拍数の増加とHRVの低下,EDAの急激な上昇が観察され,恐怖状態では交感神経系の活性化が顕著であることが確認された。一方,VRコンテンツに没入して集中している区間では,比較的緩やかな生体信号の変化が見られ,これは高い恐怖区間に比べて生理的覚醒水準が低く,注意集中状態であると解釈される。以上の結果をもとに開発されたリアルタイム感情推定システムは,VR環境下でユーザーの恐怖および驚き状態をリアルタイムでモニタリングし,フィードバックすることに活用できることを示唆している.
}%
\KeyWords{VR, 感情推定, 生体信号, 行動情報}%
]%%%%%%%%%%%%%%%%%%%% 1段組部分終わり


\section{はじめに}%%%%%%%%%%%%%%%%%%%%

VRコンテンツの評価は主観的であることが多く,制作者の意図した感情がユーザに伝わっているか客観的に把握することは難しい.本研究では,HMD(ヘッドマウントディスプレイ),皮膚電気活動(EDA)・心電(ECG)・筋電(EMG)の複数の生体センサを統合し,VRコンテンツ体験中のユーザの集中,恐怖,驚きといった感情を実時間で推定するフレームワークを構築する.

\section{関連研究}%%%%%%%%%%%%%%%%%%%%

生体信号を活用した感情推定はEDA・HRV・EMGなどを中心に活発に研究されてきた. EDAは交感神経活動を直接反映し、情動的覚醒度を定量化できる指標であり、HRV(心拍変異度)は自律神経系のバランス指標としてストレスや情動の極性を評価する際に主に利用される. EMGは筋収縮の電気信号を捉えることで、驚きやフリーズなどの急激な筋反応を検出する[1, 4]. 複数の生体信号を組み合わせることで単一センサーと比較して推定精度が向上し、視線や表情など非接触指標と融合して精度を高める試みも報告されている[5]. VR環境におけるリアルタイム感情分類の研究も増加傾向にある. Marín-Moralesら[2]はEEGとHRVをサポートベクターマシン(SVM)に入力し、VRシーンごとの覚醒度と情動価値(valence)を同時に分類した. Orozco-Moraら[3]はVRゲームプレイ中に心拍数をモニタリングし、動的難易度調整(Dynamic Difficulty Adjustment)アルゴリズムを実装、ストレスレベルを一定範囲内に維持することで恐怖や興奮体験を最適化した. これらの先行研究は、VRコンテンツ内での生体信号に基づくフィードバックシステムがユーザー体験を向上させる可能性を示唆している.

\section{実験}%%%%%%%%%%%%%%%%%%%%

\begin{figure}[tb]
  \begin{center}
    \includegraphics*[width=50mm]{logo.png}
  \end{center}
  \vspace*{-6mm}
  \caption{センサーの装着位置.}
  \label{figure}
\end{figure}

\begin{figure}[tb]
  \begin{center}
    \includegraphics*[width=50mm]{logo.png}
  \end{center}
  \vspace*{-6mm}
  \caption{ゲームのシーン.}
  \label{figure}
\end{figure}

本実験には平均25.33歳の健康な男性6名が参加した. 実験に使用したVRコンテンツはUnityエンジンで制作されたランダム迷路型サバイバル脱出ホラーゲームであり、全5セグメントで構成されている. 各セッションごとに地形やイベントの配置がランダムに変化するため、毎回新しい恐怖体験が提供される(図2). 生体信号の測定は以下の通り行った. ECG電極(IN+/IN−)は左胸部と右側腸骨稜に装着し、EDA電極は手のひら、EMG電極(IN+/IN−間隔20mm)は前腕中央、リファレンス電極は肘の突出部に装着した. 図1は各センサーの装着位置を模式化したものである. すべての信号はBITalinoワイヤレスバイオセンサーパッケージを用いて1,000Hzで同期収集した.


\subsection{実験手順}%%%%%%%%%%%%%%%%%%%%

\begin{enumerate} 
  \item 1名の参加者がVRコンテンツ1回セッション(約5〜6分)を行う間、生体信号とVRイベントログを同時に記録した. 
  \item コンテンツ進行中にゾンビ出現や脱出などの主要イベントをタイムスタンプとして記録し、事後分析に活用した.
\end{enumerate}

\section{実験結果}%%%%%%%%%%%%%%%%%%%%

\subsection{セグメント・イベント統合統計}
\begin{table}[tb]
\caption{セグメント・イベント統合統計}
\label{table1}
\begin{center}\footnotesize
\def\arraystretch{1.1}
\begin{tabular}{|c|c|c|c|c|c|}\hline 
区分 & 数 & 恐怖 & 緊張 & 集中 & 混合 \\ \hline
セグメント & 30 & 1 & 9 & 7 & 13 \\ \hline
イベント & 19 & 3 & 7 & 4 & 4 \\ \hline
\end{tabular}
\end{center}
\vspace*{-3mm}
\end{table}

VRコンテンツを5つのセグメントとゾンビ等場イベントに分けて分析した結果を表\ref{table1}に示す.
セグメント(規則進行)は「混合」が最も多く、ゾンビ登場瞬間には「恐怖/驚き」・「緊張」比率が上昇した.

\subsection{生体信号指標と感情解釈根拠}

HRVは自律神経系均衡を反映する代表指標で、RMSSDが減少すれば交感神経優勢 → ストレス・緊張・集中、増加すれば副交感神経優勢 → 弛緩・回復状態と解釈される\cite{bib07,bib08}.EDAは交感神経に支配される外分泌汗腺活動を測定し、SCR頻度増加は情緒的覚醒(恐怖・驚きなど)が高まることを意味する\cite{bib09,bib10}.EMGは筋肉の電気活動を記録して筋緊張を表す. RMS値の急激な上昇は驚き反射(Startle)、持続的上昇は長期的筋緊張を示唆する.\cite{bib11,bib12}

3つの指標のベースライン対比変化率と方向性を適用して恐怖/驚き、緊張/警戒、集中、混合の4つの状態をリアルタイムで分類した.

\begin{figure}[tb]
  \begin{center}
    \includegraphics*[width=50mm]{logo.png}
  \end{center}
  \vspace*{-6mm}
  \caption{セグメント別生体信号変化.}
  \label{figure}
\end{figure}

\begin{figure}[tb]
  \begin{center}
    \includegraphics*[width=50mm]{logo.png}
  \end{center}
  \vspace*{-6mm}
  \caption{セグメント平均値.}
  \label{figure}
\end{figure}

\subsection{セグメント別生体信号変化 (6名平均 ± SD)}

\begin{itemize}
\item \textbf{低覚醒セグメント(1–2):} HRV変化 −12 ± 5\%, EDA +8 ± 4\%, EMG +3 ± 6\%
\item \textbf{高覚醒セグメント(4–5):} HRV −44 ± 7\%, EDA +148 ± 22\%, EMG +37 ± 18\%
\end{itemize}

VRコンテンツを5つのセグメントに分けて各区間の平均HRV、EDA、EMG値およびベースライン対比変化率を整理した. 全般的に恐怖演出があるセグメントでHRVが減少し、EDAおよびEMG値が上昇する傾向が現れた. セグメント別生体信号変化は図3に棒グラフで提示した. HRV・EDA・EMG 3つの指標のベースライン対比変化率を一目で比較でき、セグメント4–5で交感神経活性(EDA ↑)とHRV減少が同時に極大化されることを視覚的に確認できる. また図4は各セグメント平均値を時系列で表し、コンテンツ進行に伴って覚醒が段階的に増加する様相を示す.

\subsection{ゾンビ登場 ±5秒生理反応 (19イベント平均)}

\begin{figure}[tb]
  \begin{center}
    \includegraphics*[width=50mm]{logo.png}
  \end{center}
  \vspace*{-6mm}
  \caption{ゾンビ登場 ±5秒生理反応.}
  \label{figure}
\end{figure}


最も強力な恐怖刺激であるゾンビ登場イベント前後の生体信号変化を比較した.イベント発生直前5秒区間と直後5秒区間の平均を計算した結果:

\begin{itemize}
\item \textbf{HR:} +15 ± 6\%
\item \textbf{RMSSD:} −44 ± 8\%
\item \textbf{EDA:} +148 ± 30\%
\item \textbf{EMG:} +37 ± 25\%
\end{itemize}

個人差があるが全参加者で同一方向変化を示した.

ゾンビ登場後、参加者の心拍数(HR)は平均約15\%増加し、EDAは平均30\%以上上昇した. 一方、HRV(RMSSD)はイベント後急激に減少(-20\%水準)し、心臓拍動がより規則的に変わることを示した. このような変化は驚き刺激に対する急激な交感神経反応を表し、恐怖状況で心拍急増 \& HRV低下と発汗増加が同時に発生することを確認させる. EMGの場合、ゾンビ登場瞬間一部参加者で前腕筋EMG信号の頂点が観察され、驚く筋反応を捉えた.

\section{結果解釈および考察}%%%%%%%%%%%%%%%%%%%%

HRV減少 \& EDA上昇はVR恐怖刺激の代表的交感神経指標であることを再確認した. EMG反応は個人別にスタートル型・硬直型に二分化され、リアルタイムシステム設計時多重指標融合が必須的である. セグメント(予測可能な威嚇)とゾンビイベント(急性刺激)は定性的反応パターンが異なり、反復暴露による感作(8番)・慣れ化(5番)様相も確認された. これはゲーム難易度適応や治療的暴露でリアルタイムHRV+EDAトリガーが活用できることを示唆する.

\subsection{HRV指標の減少}

恐怖刺激が発生した区間でHRV指標が有意に減少したことは、ストレス反応の生理的特徴として解釈される. HRV減少は交感神経系が優勢になり心拍リズムが規則的で速く変化する現象を反映し、これは恐怖状況で心臓が緊張により更に規則的に打つ結果と見ることができる. このようなHRV減少現象は既存研究(Kreibig, 2010等)でも高い覚醒度の否定的感情(恐怖等)で共通的に報告されており、本研究結果がこれを再確認してくれる.

\subsection{EDA反応の個人差}

EDA信号は全般的に恐怖刺激時上昇したが、個人差が大きい指標でもあった. したがって絶対的なEDA値よりはベースライン対比変化率が感情反応評価により有意であることが判明した. 一部参加者の場合、実験室環境温度や体質的要因により初期EDA数値が高く現れたが、例えば参加者5番は開始時から汗が出ていてEDA値が高かったが、恐怖刺激後baseline対比82\%上昇を見せて最も大きな変化を記録した. 一方、平素発汗分泌が少ない参加者は恐怖状況でもEDA変化幅が相対的に小さかった. このように人ごとに異なるEDA反応傾向を見せるため、個人別補正と変化率指標活用が重要であることを確認した.

\subsection{EMG信号と驚き反応}

EMGの場合、驚き刺激時大部分の参加者で短期頂点信号(spike)が現れ、驚く筋肉反応をよく捉えた. 特に前腕筋EMGはVRコントローラーを握った手の微細な痙攣や力を入れる変化を感知し驚き反射を明らかにした. ただし一部参加者の場合、驚き瞬間に筋肉活動がむしろ減少または停止するパターンも観察された. これは極度の恐怖を感じる時「体が凍りつく」凍結反応として解釈できる. すなわち、ある人は驚く時身をすくめて筋肉活動が急騰するが、また別の場合には瞬間的に体が固まってEMG変化がそれほど大きく現れないことがある. このような個人別恐怖反応差異は今後分析でより深く考慮されるべきである.

\section{結論および今後の課題}%%%%%%%%%%%%%%%%%%%%

本研究はランダム迷路型VRホラー体験でHRV、EDA、EMG指標が恐怖・驚き刺激に敏感に反応することを確認した. 今後には標本数拡大、長期暴露適応分析、EEG・呼吸等追加センサー融合を通じて精密度を高め、リアルタイムフィードバックシステムに拡張する予定である.

%%% この\newpageは,最終ページの左右カラム高さを手動で調整する場合に挿入します.
%%% balanceパッケージを使用すれば自動的に左右の高さを揃えられます.
%\newpage

%%% balance.styを使用する場合,最後の\sectionの直前に入れます.
\balance

%%%%%%%{参考文献}%%%%%%%%%%%%%%%%%%%%%%%%%%%%%%%%%%%%%%%%%%%%%%%%%%%%%%%%%%%%%%

\begin{thebibliography}{12}

\bibitem{bib01}
J. Guixeres et al.: 
Assessing Virtual Reality Experiences with Physiological, Behavioral and Subjective Measures,
Frontiers in Psychology, Vol.~11, pp.~1157, 2020.

\bibitem{bib02}
J. Marín‑Morales et al.: 
Affective Computing in Virtual Reality: Emotion Recognition from Brain and Heartbeat Dynamics using Wearable Sensors,
Sensors, Vol.~18, No.~10, pp.~3306, 2018.

\bibitem{bib03}
C. E. Orozco‑Mora et al.: 
Real-Time Emotion Recognition Based on Physiological Signals for VR Gaming Applications,
Electronics, Vol.~13, No.~12, pp.~2324, 2024.

\bibitem{bib04}
M. Glancy \& C. S. Ang: 
The Role of Physiological Measures in Virtual Reality Gaming,
Proceedings of the ACM on Interactive, Mobile, Wearable and Ubiquitous Technologies, Vol.~5, No.~4, pp.~178, 2021.

\bibitem{bib05}
小川健一,杉本泰治:
VR環境における生体信号を用いた感情推定手法の検討,
日本バーチャルリアリティ学会論文誌,Vol.~19, No.~1, pp.~61--70, 2014.

\bibitem{bib06}
J. A. Russell: 
A circumplex model of affect,
Journal of Personality and Social Psychology, Vol.~39, No.~6, pp.~1161--1178, 1980.

\bibitem{bib07}
Task Force of The European Society of Cardiology \& The North American Society of Pacing and Electrophysiology: 
Heart Rate Variability: Standards of Measurement, Physiological Interpretation, and Clinical Use,
Circulation, Vol.~93, No.~5, pp.~1043--1065, 1996.

\bibitem{bib08}
G. G. Berntson et al.: 
Heart rate variability: origins, methods, and interpretive caveats,
Psychophysiology, Vol.~34, No.~6, pp.~623--648, 1997.

\bibitem{bib09}
W. Boucsein: 
Electrodermal Activity, 2nd ed., 
Springer, 2012.

\bibitem{bib10}
M. E. Dawson, A. M. Schell \& D. L. Filion: 
The electrodermal system, 
in Handbook of Psychophysiology, Cambridge University Press, pp.~159--181, 2007.

\bibitem{bib11}
J. T. Cacioppo et al.: 
The psychophysiology of emotion,
in Mind and Body, pp.~97--124, 1986.

\bibitem{bib12}
P. J. Lang, M. M. Bradley \& B. N. Cuthbert: 
Emotion, attention, and the startle reflex,
Psychological Review, Vol.~97, No.~3, pp.~377--395, 1990.

\end{thebibliography}

\end{document}
