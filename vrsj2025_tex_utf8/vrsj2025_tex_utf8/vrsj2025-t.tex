%%%%%%%%%%%%%%%%%%%%%%%%%%%%%%%%%%%%%%%%%%%%%%%%%%%%%%%%%%%%%%%%%%%%%%
%  日本バーチャルリアリティ学会大会論文集
%  大会論文集投稿用原稿執筆要領(サンプル原稿)
%
%  Apr. 10, 2013  Arranged by Megumi Nakao
%  Feb.  5, 2014  Arranged by Keita Suzuki, Aichi Institute of Technology
%  Feb. 20, 2015  Arranged by Yasuyuki Yanagida
%  Feb. 20, 2019  Arranged by Shoichi Hasegawa
%%%%%%%%%%%%%%%%%%%%%%%%%%%%%%%%%%%%%%%%%%%%%%%%%%%%%%%%%%%%%%%%%%%%%%

\documentclass[a4paper]{jarticle}
%%% 本大会論文集固有のパラメータ,定義を読み込み.
  \usepackage{vrsjj}
%%% 図貼り付け用.必要に応じて,使用環境に適合するよう編集してください.
  \usepackage[dvipdfmx]{graphicx}
%%% 最終ページの高さを自動的に揃える場合,balanceパッケージを使用可.
%%% multicolパッケージを使うと脚注が二段組でなくなるため,
%%% 脚注の仕組みを利用している英文著者の表示と干渉します.
  \usepackage{balance}

  \special{pdf: pagesize width 210truemm height 297truemm} 
  
%%% ヘッダ用定義.
\newcounter{vrsjyear}
\newcounter{vrsjmonth}
\newcounter{vrsjnum}

\setcounter{vrsjyear}{2025}
\setcounter{vrsjmonth}{9}
\setcounter{vrsjnum}{\value{vrsjyear}}
\addtocounter{vrsjnum}{-1995}

%%% 行間の指定: \baselinestrechの値が1.32で1ページ50行に相当します.
\renewcommand{\baselinestretch}{1.32}

\begin{document}%%%%%%%%%%%%%%%%%%%%%%%%%%%%%%%%%%%%%%%%%%%%%%%%%%%%%%
\small %%% フォントのサイズを small (9 pt) に設定.
\twocolumn[%%%%%%%%%%%%%%%%%%%% []内が1段組部分.
%%%【必須】
\HeadComm{This article is a technical report without peer review, and its polished and/or extended version may be published elsewhere.}%
%%% 【必須】ロゴとヘッダ.変更しないでください.
\ProcTitle{第\arabic{vrsjnum}回日本バーチャルリアリティ学会大会論文集(\arabic{vrsjyear}年\arabic{vrsjmonth}月)}%
%%%
%%% 【必須】和文タイトル.
\JTitle{VRコンテンツ体験時における生体・行動情報の統合に基づく\\リアルタイム感情推定フレームワーク}%
%%% 【オプション】英文タイトル.英文タイトルを記載する場合のみ有効にしてください.
\ETitle{A Real-Time Emotion Estimation Framework Based on Integration of Physiological and Behavioral Information During VR Content Experience}
%%% 【必須】和文著者.
\JAuthor{趙 聖化$^{1)}$}%
%%% 【オプション】英文著者.
\EAuthor{Sunghwa CHO}
%%% 【必須】和文所属.機関名と住所の間の改行はなくなりました
\Affiliation{1) 関西学院大学 人間システム工学科 井村研究室 (〒669-1330 兵庫県三田市学園上ケ原1番1号, fqd69170@kwansei.ac.jp)}%
%%% 【必須】和文要旨
\Abstract{%
本研究では,Unityで制作したランダム迷路型VRホラーゲームを用いて,体験者の生体信号の変化を統計的に分析し,集中・恐怖・驚き状態の生理的特徴を明らかにした.健康な男性6名(平均25.33歳)が参加し,手のひらのEDA,左胸部‐腸骨稜のECG,前腕筋EMG(電極間20mm)を1kHzで測定しながら,約5〜6分間ゲームをプレイした.セグメントごとにゾンビが出現した直後の平均HRV(RMSSD)は安静時と比べて−44%,EDAは+148%,EMGは+37%上昇し,恐怖刺激が交感神経系の活性化および筋緊張の増加を引き起こすことが確認された. 結果分析では,VRの恐怖刺激時に心拍数の増加とHRVの低下,EDAの急激な上昇が観察され,恐怖状態では交感神経系の活性化が顕著であることが確認された。一方,VRコンテンツに没入して集中している区間では,比較的緩やかな生体信号の変化が見られ,これは高い恐怖区間に比べて生理的覚醒水準が低く,注意集中状態であると解釈される。以上の結果をもとに開発されたリアルタイム感情推定システムは,VR環境下でユーザーの恐怖および驚き状態をリアルタイムでモニタリングし,フィードバックすることに活用できることを示唆している.
}%
\KeyWords{VR, 感情推定, 生体信号, 行動情報}%
]%%%%%%%%%%%%%%%%%%%% 1段組部分終わり


\section{はじめに}%%%%%%%%%%%%%%%%%%%%

VRコンテンツの評価は主観的であることが多く,制作者の意図した感情がユーザに伝わっているか客観的に把握することは難しい.本研究では,HMD(ヘッドマウントディスプレイ),皮膚電気活動(EDA)・心電(ECG)・筋電(EMG)の複数の生体センサを統合し,VRコンテンツ体験中のユーザの集中,恐怖,驚きといった感情を実時間で推定するフレームワークを構築する.

\section{関連研究}%%%%%%%%%%%%%%%%%%%%

生体信号を活用した感情推定はEDA・HRV・EMGなどを中心に活発に研究されてきた. EDAは交感神経活動を直接反映し、情動的覚醒度を定量化できる指標であり、HRVは自律神経系のバランス指標としてストレスや情動の極性を評価する際に主に利用される. EMGは筋収縮の電気信号を捉えることで、驚きやフリーズなどの急激な筋反応を検出する[1, 4]. 複数の生体信号を組み合わせることで単一センサーと比較して推定精度が向上し、視線や表情など非接触指標と融合して精度を高める試みも報告されている[5]. VR環境におけるリアルタイム感情分類の研究も増加傾向にある. Marín-Moralesら[2]はEEGとHRVをサポートベクターマシン(SVM)に入力し、VRシーンごとの覚醒度と情動価値(valence)を同時に分類した. Orozco-Moraら[3]はVRゲームプレイ中に心拍数をモニタリングし、動的難易度調整(Dynamic Difficulty Adjustment)アルゴリズムを実装、ストレスレベルを一定範囲内に維持することで恐怖や興奮体験を最適化した. これらの先行研究は、VRコンテンツ内での生体信号に基づくフィードバックシステムがユーザー体験を向上させる可能性を示唆している.

\section{実験}%%%%%%%%%%%%%%%%%%%%
本実験には平均25.33歳の健康な男性6名が参加した. 実験に使用したVRコンテンツはUnityエンジンで制作されたランダム迷路型サバイバル脱出ホラーゲームであり、全5セグメントで構成されている. 各セッションごとに地形やイベントの配置がランダムに変化するため、毎回新しい恐怖体験が提供される(図2). 生体信号の測定は以下の通り行った. ECG電極(IN+/IN−)は左胸部と右側腸骨稜に装着し、EDA電極は手のひら、EMG電極(IN+/IN−間隔20mm)は前腕中央、リファレンス電極は肘の突出部に装着した. 図1は各センサーの装着位置を模式化したものである. すべての信号はBITalino(PLUX)ワイヤレスバイオセンサーパッケージを用いて1,000Hzで同期収集した.

\subsection{実験手順}%%%%%%%%%%%%%%%%%%%%
1. 1名の参加者がVRコンテンツ1回セッション(約5〜6分)を行う間、生体信号とVRイベントログを同時に記録した. 
2. コンテンツ進行中にゾンビ出現や脱出などの主要イベントをタイムスタンプとして記録し、事後分析に活用した.

\subsection{実験結果}%%%%%%%%%%%%%%%%%%%%



\section{各部分のレイアウトとフォントについて}%%%%%%%%%%%%%%%%%%%%

本サンプルファイルは,日本語\LaTeXe の標準的なクラスである\verb+jarticle+を使用し,
本大会予稿集用独自のパラメータや定義を規定する\verb+vrsjj.sty+を読み込んでいます.

\subsection{タイトル部}

タイトル部は,例のように1段組としてください.
1ページ目の左上には,
{\bf 日本バーチャルリアリティ学会のロゴマーク(VRSJロゴ)}を
縦1 cm×横2 cm程度の大きさになるように貼り付けてください.
VRSJロゴについては,日本バーチャルリアリティ学会ホームページ
 (http://www.vrsj.org/) 等をご参照ください.
また,1ページのヘッダのみに,例のように右に詰めて,
「{\bf 第\arabic{vrsjnum}回日本バーチャルリアリティ学会大会論文集(\arabic{vrsjyear}年\arabic{vrsjmonth}月)}」と
{\bf ゴシック体9pt.}を用いて記入してください.
また,
1p目のヘッダに「This article is a technical report without peer review, and its polished and/or extended version may be published elsewhere.」と{\footnotesize Times 8 pt}で記入してください.

ヘッダより1行あけてタイトルを記述してください.
タイトルは{\bf ゴシック体18 pt}を用い,センタリングにしてください(\verb+\JTitle+を使用).

英文タイトルは任意です.付けない場合は,タイトル著者名ともコメントアウトしてください.付ける場合は,
%2015年より英文タイトルは任意でしたが,2019年大会では国際会議ICAT-EGVEと併催することから英文タイトルを必須としました.%
{\normalsize Times 10 pt}のフォントを用い,前置詞以外の単語の先頭は大文字で,
センタリングにしてください(\verb+\ETitle+を使用).

1行あけて,例のように著者名を{\normalsize 明朝体10 pt}を用いて記述し,
センタリングにしてください(\verb+\JAuthor+を使用).
英文の著者名は日本語の著者名の下にTimes 10 ptを用いて記入してください(\verb+\EAuthor+を使用).

1行あけて著者の所属を明朝体9 ptを用いて記入し,センタリングにしてください(\verb+\Affiliation+を使用).
複数の著者の所属が異なる場合には,
例のように著者名に付けた片カッコ付き数字を添えて記入してください.

1行あけて和文概要を,明朝体 9 ptを用いて記入してください(\verb+\Abstract+を使用).
`\textgt{概要}'という文字は\textgt{ゴシック体}にします.
左右のインデントは明朝体9 ptで5文字程度になるようにしてください.
次の行に3-4個の和文キーワードを例のように明朝体9ptにて記入してください.
`\textgt{キーワード}'という文字は\textgt{ゴシック体}にします.

\subsection{本文部分}

キーワードの後,2行あけて本文に移ります.
本文は横2段組(本サンプルファイルでは\verb+\twocolumn+を使用),
50行(行間約14.4pt),明朝体9 ptで作成してください.
段落冒頭は1文字分字下げします.

%% 見出しが複数行にわたる場合,見出し内の行間に本文の行間倍率が適用されると間延びするので,
%% ここだけ行間設定を変更しています.
\begingroup
\renewcommand{\baselinestretch}{0.9}
\section{見出し(見出しが複数行に渡る場合には,このようにインデントを付ける)}
\endgroup

\subsection{章の見出し}%%%%%%%%%%%%%%%%%%%%

見出しのレベルは3段階とし,
第1レベル(章)は,上に1行あけて{\bf ゴシック体10 pt}により
「{\bf 3. 章の見出し}」のように記入してください(\verb+\section+を使用).

\subsection{節の見出し}

第2レベル(節)の見出しは前後に空白行を設けず,
{\bf ゴシック体9 pt}により「{\bf 3.2 節の見出し}」のように記入してください(\verb+\subsection+を使用).

\subsubsection{項の見出し}

第3レベル(項)の見出しも前後に空白行を設けず,
{\bf ゴシック体9 pt}により「{\bf 3.2.1 項の見出し}」のように記入してください(\verb+\subsubsection+を使用).

\section{数式および数学記号}%%%%%%%%%%%%%%%%%%%%

数式はセンタリングし,式番号はカッコ付きの通し番号で右詰めとしてください.
\begin{equation}
  F(x) = \frac{a}{\sqrt{a+b}}\int_a^b{g(t)}dt
\end{equation}
また,数式の前後には半行$\sim$1行程度の空白を設けてください.(\LaTeX の場合,
通常は\verb+equation+環境などの使用により自動的に挿入されます.)

\section{図表}%%%%%%%%%%%%%%%%%%%%

図表は,図\ref{figure}のように,
{\bf 本文中で引用したページの四隅いずれかまたは上下端}に置くことを推奨します.
可読性の観点から,本文中の引用箇所より前のページに図を置くことや,
原稿末尾にまとめて置くことは避けてください.
図の前後には空白行を1行設け,図のキャプションは図の下に,表のキャプションは表の上に置いてください.
図番号,表番号は通し番号とし,{\bf ゴシック体9 pt}で記入してください.

%%% この\newpageは,最終ページの左右カラム高さを手動で調整する場合に挿入します.
%%% balanceパッケージを使用すれば自動的に左右の高さを揃えられます.
%\newpage

\section{最終ページのレイアウトと参考文献}

最後のページは左右の段落ができるだけそろうように調整してください.
手動で調整する場合,適切な場所に\verb+\newpage+を挿入してください.
自動的に調整する場合,\verb+balance+パッケージを使用できます.
最後の\verb+\section+の直前に\verb+\balance+を入れると,自動的に調整されます.

参考文献は出現順に番号を付け,該当個所に\cite{bib01}\cite{bib02}\cite{bib03}\cite{bib04}のようにカギカッコで指示してください.
参考文献の引用リストは例を参考にして,文末に1行あけ,
{\normalsize\bf ゴシック体10 pt}センタリングで「{\normalsize\bf 参考文献}」と記した後に,番号順に記入してください.
姓名の記法や誌名巻号の略記法など形式について厳密な指定はありませんが,リストの中で統一を取るようにしてください.

\section{PDF出力}
原稿はPDFにて投稿してください.投稿前に,
\begin{itemize}
 \item 書式の乱れがないか
 \item 参照文献番号があっているか
 \item 画像は印刷に耐えうるクオリティか
 \item A4サイズになっているか
 \item 白紙ページがないか
 \item 規定ページ数に収まっているか
\end{itemize}
などを必ず確認してください.
また,出版プロセスでPDFを加工(ノンブル付与)するため,PDFのセキュリティなどは解除いただく様お願いいたします.

%%% balance.styを使用する場合,最後の\sectionの直前に入れます.
\balance

\section{むすび}
なお,この原稿は\LaTeXe を用いて作成したものです.
この原稿は本執筆要領に基づいて作成されたサンプル原稿の一つであり,
本スタイルファイルを使用する義務は全くありません.
また,本スタイルファイルを使用することで発生する
いかなる不具合についても対処することはできません.
\\

\begin{figure}[tb]
  \begin{center}
    \includegraphics*[width=50mm]{logo.png}
  \end{center}
  \vspace*{-6mm}
  \caption{図のキャプションは図の下に置く.図中のテキストはキャプションと同じかやや小さくなるように調整すること.}
  \label{figure}
\end{figure}

\begin{table}[tb]
\caption{表のキャプションは表の上に置く}
\label{table}
\begin{center}\small
\def\arraystretch{1.2}
\begin{tabular}{|c||c|c|}\hline 
No. & Real & Estimated \\ \hline \hline
1   & 1.5  & 1.2       \\ \hline
2   & 2.5  & 2.3       \\ \hline
3   & 3.5  & 3.4       \\ \hline
\end{tabular}
\end{center}
\vspace*{-3mm}
\end{table}

\noindent{\bf 謝辞} 謝辞は結論の後に書いてください.\\

\noindent{\bf 付録} 付録は参考文献の前に書いてください.


%%%%%%%{参考文献}%%%%%%%%%%%%%%%%%%%%%%%%%%%%%%%%%%%%%%%%%%%%%%%%%%%%%%%%%%%%%%

\begin{thebibliography}{9}

\bibitem{bib01}
バーチャル太郎,現実花恵:
日本バーチャルリアリティ学会大会論文集の書き方,
日本バーチャルリアリティ学会第1回大会論文集,
pp.~1--2, 1996.

\bibitem{bib02}
バーチャル太郎,現実花恵:
日本バーチャルリアリティ学会投稿論文の書き方,
日本バーチャルリアリティ学会論文誌,
Vol.~1, No.~2, pp.~201--206, 1996.

\bibitem{bib03}
バーチャル太郎,現実花恵:
日本バーチャルリアリティ学会解説の書き方,
日本バーチャルリアリティ学会誌,
Vol.~2, No.~4, pp.~11--16, 1997.

\bibitem{bib04}
人工現太郎,実 感子:
日本バーチャルリアリティ学会大会論文の書き方,
日本バーチャルリアリティ学会大会論文集,
Vol.~4, pp.~1--2, 1999.

\end{thebibliography}

\end{document}
